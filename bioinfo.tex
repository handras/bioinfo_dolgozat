%% Based on a TeXnicCenter-Template by Gyorgy SZEIDL.
%%%%%%%%%%%%%%%%%%%%%%%%%%%%%%%%%%%%%%%%%%%%%%%%%%%%%%%%%%%%%

%------------------------------------------------------------
%
\documentclass[a4paper,12pt,leqno, notitlepage]{article}%
%Options -- Point size:  10pt (default), 11pt, 12pt
%        -- Paper size:  letterpaper (default), a4paper, a5paper, b5paper
%                        legalpaper, executivepaper
%        -- Orientation  (portrait is the default)
%                        landscape
%        -- Print size:  oneside (default), twoside
%        -- Quality      final(default), draft
%        -- Title page   notitlepage, titlepage(default)
%        -- Columns      onecolumn(default), twocolumn
%        -- Equation numbering (equation numbers on the right is the default)
%                        leqno
%        -- Displayed equations (centered is the default)
%                        fleqn (equations start at the same distance from the right side)
%        -- Open bibliography style (closed is the default)
%                        openbib
% For instance the command
%           \documentclass[a4paper,12pt,leqno]{article}
% ensures that the paper size is a4, the fonts are typeset at the size 12p
% and the equation numbers are on the left side
%
 
\setlength{\parindent}{2.2em}
\setlength{\parskip}{1.1em}
\renewcommand{\baselinestretch}{1.1}
\usepackage{amsmath}%
\usepackage{amsfonts}%
\usepackage{amssymb}%
\usepackage{graphicx}
\usepackage[utf8]{inputenc}
\usepackage[magyar]{babel}
%-------------------------------------------
\frenchspacing
\sloppy

% define references
\newcommand{\figref}[1]{\ref{fig:#1}.}
\renewcommand{\eqref}[1]{(\ref{eq:#1})}
\newcommand{\listref}[1]{\ref{listing:#1}.}
\newcommand{\sectref}[1]{section \ref{sect:#1}-\nameref{sect:#1}}
\newcommand{\tabref}[1]{\ref{tab:#1}.}

\begin{document}

\title{bioinfo dolgozat}
\author{Frank Dániel \\ Széchenyi István Egyetem, Győr
				\and
				Holczbauer Bálint \\ Széchenyi István Egyetem, Győr
				\and
				Horváth András \\ Széchenyi István Egyetem, Győr}
\date{2020 május 1.}
\maketitle

\begin{abstract}

Ez a dolgozat a Bevezetés a bioinformatikába tárgy keretén belül készült.
Célja a \emph{Phylogenetic Diversity and the Greedy Algorithm}\cite{ptrrees_and_greedy} című tanulmány feldolgázsa és az eredmények bemutatása.
A dolgozatban minden felmerülő definíciót tisztázunk és értelmezünk. Az olvasónak nem szükséges a szakterülezhez kapcsolódó előismeretekkel rendelkeznie.

\end{abstract}

\section{Szoftver\-fejlesztési módszerek}
\label{sec:s}

\bibliography{biblio}{}
\bibliographystyle{plain}

%\begin{thebibliography}{1}
% Copy generated biblography here if you heve problem with accented characters
%\end{thebibliography}


\end{document}
