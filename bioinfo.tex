%% Based on a TeXnicCenter-Template by Gyorgy SZEIDL.
%%%%%%%%%%%%%%%%%%%%%%%%%%%%%%%%%%%%%%%%%%%%%%%%%%%%%%%%%%%%%

%------------------------------------------------------------
%
\documentclass[a4paper,12pt,leqno, notitlepage]{article}%
%Options -- Point size:  10pt (default), 11pt, 12pt
%        -- Paper size:  letterpaper (default), a4paper, a5paper, b5paper
%                        legalpaper, executivepaper
%        -- Orientation  (portrait is the default)
%                        landscape
%        -- Print size:  oneside (default), twoside
%        -- Quality      final(default), draft
%        -- Title page   notitlepage, titlepage(default)
%        -- Columns      onecolumn(default), twocolumn
%        -- Equation numbering (equation numbers on the right is the default)
%                        leqno
%        -- Displayed equations (centered is the default)
%                        fleqn (equations start at the same distance from the right side)
%        -- Open bibliography style (closed is the default)
%                        openbib
% For instance the command
%           \documentclass[a4paper,12pt,leqno]{article}
% ensures that the paper size is a4, the fonts are typeset at the size 12p
% and the equation numbers are on the left side
%
 
\setlength{\parindent}{2.2em}
\setlength{\parskip}{1.1em}
\renewcommand{\baselinestretch}{1.1}
\usepackage{amsmath}%
\usepackage{amsthm}%
\usepackage{amsfonts}%
\usepackage{amssymb}%
\usepackage{graphicx}
\usepackage[utf8]{inputenc}
\usepackage[magyar]{babel}
%-------------------------------------------
\frenchspacing
\sloppy

% define references
\newcommand{\figref}[1]{\ref{fig:#1}.}
\renewcommand{\eqref}[1]{(\ref{eq:#1})}
\newcommand{\listref}[1]{\ref{listing:#1}.}
\newcommand{\sectref}[1]{section \ref{sect:#1}-\nameref{sect:#1}}
\newcommand{\tabref}[1]{\ref{tab:#1}.}
\newcommand{\defref}[1]{\ref{def:#1}.}


\newcommand{\todo}[1]{\par\texttt{\textbf{todo: #1}}\par}

% define theorems
\theoremstyle{plain}
\newtheorem{thm}{Theorem}[section] % reset theorem numbering for each section

\theoremstyle{definition}
\newtheorem{defn}[thm]{Definíció} % definition numbers are dependent on theorem numbers
\newtheorem{exmp}[thm]{Példa} % same for example numbers

\begin{document}

\title{bioinfo dolgozat}
\author{Frank Dániel \\ Széchenyi István Egyetem, Győr
				\and
				Holczbauer Bálint \\ Széchenyi István Egyetem, Győr
				\and
				Horváth András \\ Széchenyi István Egyetem, Győr}
\date{2020 május 1.}
\maketitle

\begin{abstract}

Ez a dolgozat a Bevezetés a bioinformatikába tárgy keretén belül készült.
Célja a \emph{Phylogenetic Diversity and the Greedy Algorithm}\cite{ptrrees_and_greedy} című tanulmány feldolgázsa és az eredmények bemutatása.
A dolgozatban minden felmerülő definíciót tisztázunk és értelmezünk. Az olvasónak nem szükséges a szakterülezhez kapcsolódó előismeretekkel rendelkeznie.

\end{abstract}

\section{Alapismeretek áttekintáse}
\label{sec:alapok}

A dolgozat elején összefoglalót adunk a dolgozatban felmerülő, a megértéshez szükséges matematikai illetve biológia  rendszertani ismeretekről.

\subsection{Gráfelméleti alapismeretek}
\label{grafok}

Ez a szakasz a Fleiner Tamás \emph{A számítástudomány alapjai}\cite{nesz} összefoglalója felhasználásával készült.

\begin{defn}\label{def:graf}
A $G=(V, E)$ pár egy egyszerű gráf, ha (1) $V \neq\emptyset$ és (2) $E \subseteq{V \choose 2} := \{\{u, v\} : u, v \in V, u \neq v\}$
, azaz E elemei V bizonyos kételemű részhalmazai.
Ha G egy gráf, akkor $V (G)$ jelöli G csúcsainak, $E(G)$ pedig $G$ éleinek halmazát, azaz $V (G)$ az a $V$ halmaz, és $E(G)$ az az E halmaz, amire $G = (V, E)$. A $G$ egyszerű gráf véges, ha $V$ véges halmaz.
\end{defn}

\begin{defn}\label{def:fa}
A $G$ gráf fa, amennyiben összefüggő (azaz minden csúcsa között vezet út), és nem tartalmaz kört.
\end{defn}

\subsection{Redszertani alapismeretek}
\label{rendszertani}

A \emph{filogenetikai fa} (vagy más neveken \emph{evolúciós fa} vagy \emph{dendogram}) a biológiában arra használatos, hogy fajok közötti evolúciós kapcsolatokat szemléltessen.
Az evolúciós fa megalkotása során a fán feltüntetett biológiai egységek között megfigyelt fizikális illetve genetikai hasonlóságokat és különbözőségeket veszünk figyelembe.
Alapvetően megkülönböztetjük a \emph{gyökeres} illetve a \emph{gyökérmentes} filogenetikai fát.
\todo{Példa fa keresése}

Az előző fejezetben ismertetett matematikai fogalmak segítségével megadhatjuk a filogenetikai fa formális definícióját.
 
\begin{defn}\label{def:fgfa}
A $\tau$ (gyökérmentes) filogenetikai X-fa egy olyan $(\T, \phi)$ rendezett pár, ahol $T=(V_\T,E_T)$ egy olyan fa amiben nincs két fokú csúcs, továbbá a $\phi$ egy bijekció $X$ről $T$ levélhalmazára.
\end{defn}

\bibliography{biblio}{}
\bibliographystyle{plain}

%\begin{thebibliography}{1}
% Copy generated biblography here if you heve problem with accented characters
%\end{thebibliography}


\end{document}
